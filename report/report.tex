\documentclass[11pt]{report}

\usepackage[toc,page]{appendix}
% CPS equations
\usepackage{amsmath}
\usepackage{stmaryrd}
% Grammar
\usepackage{syntax}
% Stages diagram
\usepackage{tikz}
% Stages diagram elements
\usepackage{graphicx}
% Chapter number and name on one line
%\usepackage{titlesec}
%\titleformat{\chapter}[hang] 
%{\normalfont\huge\bfseries}{\thechapter}{1em}{} 
% No new page after chapter
%\usepackage{etoolbox}
%\makeatletter
%\patchcmd{\chapter}{\if@openright\cleardoublepage\else\clearpage\fi}{}{}{}
%\makeatother
% Source code listings
\usepackage{listings}

%%%%%%%%%%%%%%%%%%%%%%%%%

\title{LJSP - A LISP to asm.js compiler}
\author{Jan S\"ondermann}
\date{1st January 1900}

% Commands to make CPS equations easier to write
\newcommand{\eqdef}{\stackrel{\text{def}}{=}}%
\newcommand{\cpstrans}[1]{\ensuremath{\mathcal{K}\llbracket #1 \rrbracket}}

\begin{document}
% TODO
% why scala? why LLVM IR/C



\maketitle

\begin{center}
\textbf{Originality avowal}
\end{center}

I verify that I am the sole author of this report, except where explicitly stated to the contrary.

I grant the right to King's College London to make paper and electronic copies of the submitted work for purposes of marking, plagiarism detection and archival, and to upload a copy of the work to Turnitin or another trusted plagiarism detection service.

\begin{flushright}
Jan Söndermann \\
@@date@@@
\end{flushright}
\newpage
			
\begin{center}
\textbf{Abstract}
\end{center}

@@@
\newpage

\begin{center}
\textbf{Acknowledgements}
\end{center}
I would like to thank my supervisor Dr. Christian Urban for supervising my project and for providing me with both guidance and freedom.
\newpage

\tableofcontents
\newpage

\chapter{Introduction}
One of the main trends of today's Internet is the shift from traditionally offline or standalone programs to the web. This phenomenon, often @dubbed@ "Web 2.0"@ajax@ is visible in the success of web applications such as GMail and Facebook @other examples@.

This success of the web has brought with it the rise of the programming language it is build on: JavaScript, the only language that is supported by all major browsers. Statistics, such as the number of repositories created on GitHub (a popular code sharing website) using JavaScript, show this success: in 2013, JavaScript led this list by a substantial margin. @cite

Since JavaScript is no longer confined to being used for simple animation and input-validation tasks, the complexity of web apps has increased tremendously. Modern JavaScript frameworks, such as Ember.js@link or Meteor@link use the full register of functionality the language offers and are @just as complex as traditional web frameworks that use languages such as Ruby or Python.@

Unfortunately, this proliferation of JavaScript has taken the language far beyond the tasks it was initially conceived for @cite. There is a widespread consensus@cite the language has a number of shortcomings. Brendan Eich, the creator of JavaScript, writes\cite{brendeich} about its early history:
\begin{quote}
In April 1995 I joined Netscape in order to "add Scheme to the browser." [...]

So in 10 days in May 1995, I prototyped "Mocha," the code name Marc Andreessen had chosen. [...]

To overcome all doubts, I needed a demo in 10 days. I worked day and night, and consequently made a few language-design mistakes (some recapitulating bad design paths in the evolution of LISP), but I met the deadline and did the demo.
\end{quote}

Criticism of JavaScript has mostly focussed on two areas:
\begin{itemize}
\item Inconsistencies and mistakes in the design of the language. Examples for this often given include the confusion around JavaScript's large number of falsy values (\texttt{0},  \texttt{''}, \texttt{NaN}, \texttt{false}, \texttt{null} and \texttt{undefined}) and its set of reserved keywords, which includes a large number of words not in use by the language but doesn't include \texttt{NaN} and \texttt{undefined}. This makes it necessary to test for \texttt{undefined}ness using \\
\mbox{\texttt{typeof x === 'undefined'}} to avoid comparing against a redefined \texttt{undefined}.
\item Slow execution speed. Writing fast interpreters for JavaScript has been an extraordinary difficult task for browser @manufacturers. @why?@
\end{itemize}

Programmers have responded to the first problem in ways that include limiting themselves to a subset of JavaScript that excludes the inconsistent and badly-designed parts (cf. the very popular "JavaScript: The Good Parts"@cite). Another response has been to create new languages that compile to JavaScript, such as the very popular CoffeeScript @cite, Microsoft's TypeScript and Google's Dart.

The second problem has been partly remedied by a new generation of JavaScript engines spearheaded by Google's V8, released in 2008 as part of Google Chrome. The other browser makers, including Mozilla soon followed by rewriting their own JavaScript engines. These new engines often brought impressive speed gains.

In early 2013, Mozilla released asm.js, a project that claims to take these two approaches to their logical conclusion. It defines an extremely limited, statically typed subset of JavaScript that can be executed very quickly. This subset is intended as compilation targets for high level languages.

This project defines such a language and provides a compiler for it that outputs, among other targets, asm.js. It sets out to explore the performance@

\chapter{Background \& Existing Work}
\section{Technologies involved}
\subsection{asm.js}
\subsection{emscripten and LLVM}
\section{Similar Languages}

\chapter{Requirements}
\section{User Requirements}
\section{System Requirement}

\chapter{Specification}
\section{LJSP Grammar}
The grammar of LJSP in Extended Backus-Naur Form is given below.
\begin{grammar}
<program> ::= <defines> [ <expr> ] <defines>

<defines> ::= <define> <defines> | $\epsilon$

<define> ::= `(define (' <ident> <params> `)' <expr> `)'

<expr> ::= <double>
\alt <ident>
\alt `(if' <expr> <expr> <expr> `)'
\alt <lambda>
\alt `(let (' <letblocks> `)' <expr> `)'
\alt `(' <primOp> <args> `)'
\alt `(' <expr> <args> `)'

% add scientific form
<double> ::= \texttt{-?(\textbackslash d+(\textbackslash.\textbackslash d*)?|\textbackslash d*\textbackslash.\textbackslash d+)}

<ident> ::= \texttt{[a-zA-Z=*+/\textless\textgreater!?-][a-zA-Z0-9=*+/\textless\textgreater!?-_]*}

<lambda> ::= `(lambda (' <params> `)' <expr> `)'

<params> ::= <ident> | <ident> <params>

<letblocks> ::= <letblock> | <letblock> <letblocks>

<letblock> ::= `(' <ident> <expr> `)'

<primOp> ::= `+' | `-' | `*' | `/' | `neg'
\alt `<' | `>'
\alt `min' | `max'
\alt `sqrt'

% is an application to an empty argument list legal?
<args> ::= <expr> | <expr> <args>
\end{grammar}

The grammar for expressions used in intermediate stages is as follows:
\begin{grammar}
<expr> ::= <env>
\alt `(make-lambda' <lambda> <env> `)'
\alt `(nth' <int> <expr> `)'
\alt `(get-env' <expr> `)'
\alt `(get-proc' <expr> `)'
% should this be <expr> instead of <env>?
\alt `(hoisted-lambda' <ident> <env> `)'

<env> ::= `(make-env' <idents> `)'

<int> ::= \texttt{-?(0|[1-9][0-9]*)}

<idents> ::= <ident> <idents> | $\epsilon$
\end{grammar}

% subset of scheme
\section{Small/big-step semantics of LJSP}
\section{Backend Output}

\chapter{Design}
This chapter will describe @@@, this chapter: abstract, theoretical next chapter: concrete, details
\section{Overall architecture}
The compiler is made up of two major components: A hierarchy of classes that get instantiated to create the Abstract Syntax Tree (AST) @described in detail below@ and a number of compilation stages that perform various transformation on this AST. I will first describe the AST classes before going through all the individual stages of the compiler in detail.
% scheme is used for all internal code
\section{AST}
% class diagrams for all five sets of AST classes
The compiler includes five sets of class hierarchies that all have their own abstract superclass: One that reflects LJSP code, one for an intermediate representation used at a later stage and one for each compilation target: asm.js, C and LLVM IR.
\section{Detailed design of compilation stages}

\begin{figure}[ht]
\begin{center}
\includegraphics[scale=0.8]{stagesflowchart.eps}
\caption{Flowchart for compilation stages in LJSP}
\end{center}
\label{stagesflowchart}
\end{figure}

% based on from system F...
\subsection{Parsing}
\subsection{CPS-Translation}

This stage converts the program to continuation-passing style (CPS). Because it is the central stage of the compilation pipeline, we will first explain continuation-passing style in general before describing the particularities of the CPS-translation in the LJSP compiler.

\subsubsection{Continuation-Passing Style}
Code that is in Continuation-Passing Style gives explicit names to every intermediate computation and makes control flow explicit by never returning from a function call.\cite{sysftal}\cite{appel} To illustrate this, we show the transformation of a simple function \texttt{f}, given below, to CPS step-by-step.

\begin{lstlisting}
function f(a, b) {
  return a + b + 10
}
\end{lstlisting}

First, we give a name to the result of the additions that the function returns:

\begin{lstlisting}
function f1(a, b) {
  t1 = a + b + 10
  return t1
}
\end{lstlisting}

To conform to the requirement that every intermediate result be explicitely named, we need to break up the long sum into two computations. This makes it clear that the first addition gets precedence before the second:

\begin{lstlisting}
function f2(a, b) {
  t2 = a + b
  t1 = t2 + 10
  return t1
}
\end{lstlisting}

The last step in our example is the one that gives Continuation-Passing Style its name: Functions in CPS conform programs do not return, but instead receive a function, usually called "continuation", that the caller passes in an additional parameter. This continuation can be understood as the remainder of the program that processes the result computed by the function:

\begin{lstlisting}
function f3(cont, a, b) {
  t2 = a + b
  t1 = t2 + 10
  cont(t1)
}
\end{lstlisting}

Code that is CPS conform is nearly linear: it consists of a sequence of let expressions followed by a single function call. The only expression that violates this linearity is the if-expression that branches execution flow.

To further illustrate the concept of continuations, we give a short program written in Scheme below. Scheme is unique among most programming languages in that it gives the programmer access to the continuations in his program using a function called \texttt{call/cc}\footnote{call/cc stands for "call with current continuation". Unlike in most other languages, identifiers in Scheme can contain the '/' character.}. \texttt{call/cc} takes as its one parameter a function, which in term gets called with the current continuation as argument. Our example program uses \texttt{call/cc} to save and later reuse a continuation inside an arithmetic operation.

We first define a variable \texttt{cont} that will later hold our saved continuation. Variables in Scheme need to be defined before they can be set! later, right now \texttt{cont} holds a dummy value of 0. We also define a function \texttt{set-cont} that takes a parameter, sets \texttt{cont} to its value and returns 1. This may seem redundant, but it simplifies the next line.
\begin{lstlisting}
> (define (cont) 0)
> (define (set-cont c) (set! cont c) 1)
\end{lstlisting}

Next, we assign our continuation. \texttt{call/cc} gets called with set-cont which in turn gets called with the continuation. The return value of set-cont, which is 1, is the value that gets passed to the arithmetic operations and this line returns 3.
\begin{lstlisting}
> (+ 1 (* 2 (call/cc set-cont)))
3
\end{lstlisting}

We can now call this continuation again with a different value.
\begin{lstlisting}
> (cont 2)
5
> (cont 10)
> 21
\end{lstlisting}

One way of visualising the continuation is to write it as \texttt{(+ 1 (* 2 ...))}. This is the continuation of the code at \texttt{...} Whatever value this code computes, it passes it on to our continuation. It is worth noting that the contination is not just simply a function that multiplies its argument by 2 and adds 1, as the following code illustrates:
\begin{lstlisting}
> (+ 1000 (cont 2))
5
\end{lstlisting}

Calling our saved continuation substitutes the continuation that we see in the code above, \texttt{(+ 1000 ...)}, with our saved continuation.

\subsubsection{CPS-Translation in LJSP}
\begin{figure}[ht]
\begin{alignat*}{2}
&\cpstrans{y} k &&\eqdef k(y) \\
%
&\cpstrans{d} k &&\eqdef k(d) \\
%
&\cpstrans{(\text{if}\ e_1\ e_2\ e_3)} k &&\eqdef \cpstrans{e_1} \lambda x.(\text{if}\ x\ \cpstrans{e_2}k\ \cpstrans{e_3}k) \\
%
&\cpstrans{(\text{define}\ (name, p_1, \dots, p_n)\ e)} k &&\eqdef k((\text{define}\ (name, f_{cont}, p_1, \dots, p_n)\ \\
&&&\hspace{1cm}\cpstrans{e}f_{cont})) \\
%
&\cpstrans{(\text{lambda}\ (p_1, \dots, p_n)\ e)} k &&\eqdef k((\text{lambda}\ (f_{cont}, p_1, \dots, p_n)\ \\
&&&\hspace{1cm}\cpstrans{e}f_{cont})) \\
%
&\cpstrans{(proc\ p_1, \dots, p_n)} k &&\eqdef \cpstrans{p_1} \lambda x_1. \dots  \\
&&&\hspace{1 cm}\cpstrans{p_n} \lambda x_n.\\
&&&\hspace{1.5 cm}(proc\ \lambda x_k.k(x_k), x_1, \dots, x_n) \\
%
&\cpstrans{(prim\ p_1, p_2, \dots, p_n)} k &&\eqdef \cpstrans{(prim\ (prim\ (prim\ p_1, p_2)\ p_3) \dots p_n)} k\\
%
&\cpstrans{(prim\ p_1, p_2)} k &&\eqdef \cpstrans{p_1}\lambda x_1. \cpstrans{p_2}\lambda x_2.\\
&&&\hspace{1cm}(\text{let}\ f=(prim\ x_1, x_2)\ \text{in}\ k(f))\\
%
&\cpstrans{(\text{let}\ ((idn\ e_1))\ e_2)} k &&\eqdef \cpstrans{e_1} \lambda x_1. \\
&&&\hspace{1 cm}(\text{let}\ idn = x_1\ \text{in}\ \cpstrans{e_2} \lambda x_2.(k(x_2)))\\
\end{alignat*}
\caption{CPS translations of LJSP expressions}
\label{cpstrans}
\end{figure}

Figure~\ref{cpstrans} shows CPS translations for all LJSP expressions. In the equations, $y$ stands for a variable, $d$ for a double constant and $f$ for a fresh identifier. $\cpstrans{e} k$ stands for the CPS translation of expression $e$ with continuation $k$.

We will conclude this section on Continuation-Passing Style by describing some of the more complex CPS translation equations in more detail. The transformation for both the \texttt{define} and \texttt{lambda} expressions CPS translate the body of the function with a continuation passed to the function in a new parameter called $f_{cont}$ in the equations and \texttt{cont_n} in the code. This parameter has a correspondence in the translation for proc in the next equation: it is the $\lambda x_k.k(x_k)$ that gets added to the application when translating. The translation for primitive operations consists of two equations in the diagram, one equation to convert operations with more than one operands into multiple operations with two operands each and one equation to convert it to Continuation-Passing Style. Note that unlike the translation of regular function applications, the application of primitive operations does not include the additional continuation parameter just mentioned, as primitive operations are allowed to return.

\subsection{Closure Conversion}
\subsection{Hoisting}
\subsection{Conversion to asm.js}
\subsection{Code emission}

\chapter{Implementation}
\section{Scala}
\section{Problems encountered \& resolved}
% variables that contain functions, ftables generated at compile time
\section{Experimentation \& Optimisation}
% array index vars of type int, others of type double
% (array index vars are all generated, user doesn't see them)

\chapter{Testing}
This chapter describes the methods that were used to test the different subsystems of the LJSP compiler. While testing is imperative for all software development projects, it is especially important in compiler development. The reason for this is that it is usually impossible to determine whether or not the result given by the compiler is correct simply by looking at the generated code. \\

During the development of LJSP, two means of testing were being used:
\begin{itemize}
\item A custom testing framework written in Python that covers all of the frontend stages and some of the backend stages.
\item A ray tracer written in JavaScript that was used to test generated asm.js code specifically.
\end{itemize}
The following two sections will describe both systems in detail.

\section{Testing using \texttt{run_tests.py}}
Early on in the development of LJSP, it became necessary to test the output of every front end stage 
\section{Testing using the ray tracer}
% because asm.js evaluates to the same result regardless of
% whether or not it's compiled according to asm.js specification,
% it's possible to add invalid statements and still test the module

% ray tracer to test performance

\chapter{Evaluation}
% ast transformations would be nicer if the compiler generated the subtrees
% from short code snippets instead of creating the objects manually
\section{Evaluation against requirements}
\section{Strengths \& Weaknesses}

\chapter{Professional Issues}
% http://www.bcs.org/category/6030

\chapter{Conclusions}
\section{Future Work}
\subsection{Optimisations}
\subsection{Types}

\newpage

\chapter{Bibliography}
\begin{thebibliography}{1}
  \bibitem{sysftal} Morrisett, Greg and Walker, David and Crary, Karl and Glew, Neal (1999). {\em From System F to Typed Assembly Language}, ACM Trans. Program. Lang. Syst.
    \bibitem{brendeich} Foreword by Brendan Eich to {\em Node: Up and Running}. Available at http://chimera.labs.oreilly.com/books/1234000001808/pr02.html
    \bibitem{appel} Andrew Appel (1992). {\em Compiling with Continuations} Cambridge University Press
\end{thebibliography}


\begin{appendices}

\chapter{User Guide}
\section{Compiler}
\section{Ray Tracer}
\section{Testing}

\chapter{Class diagrams}

\chapter{Programs Listings}

\end{appendices}

\end{document}
